% descripcion.tex

\section{Descripcion del Algoritmo} \label{sec:Descripcion del Algoritmo}

El algoritmo de \textit{K vecinos más cercanos} (KNN) es como un ``vecindario'' donde cada punto de datos tiene ``vecinos'' cercanos. Este algoritmo clasifica o predice a qué grupo pertenece un nuevo punto de datos basándose en qué grupo son sus vecinos más cercanos.
\\
Imagina que tienes una serie de puntos en un mapa, algunos son rojos y otros azules. Quieres saber a qué grupo pertenece un nuevo punto. KNN mira a los K puntos más cercanos al nuevo punto y ve qué color tienen. Luego, el nuevo punto ``vota'' por el color que más ve entre sus vecinos cercanos, y así se decide a qué grupo pertenece.
\\
El número de vecinos ($K$) es importante porque determina qué tan precisa será la predicción. Si eliges un número grande de vecinos, el modelo puede ser más preciso, pero también más costoso computacionalmente. Si eliges un número pequeño de vecinos, el modelo puede ser más rápido pero menos preciso.
