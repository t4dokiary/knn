\section {Apendice} \label{sec:Apendice}
\subsection{Codigo Fuente} \label{subsec:Codigo Fuente}

El código fuente de este proyecto se encuentra disponible en el siguiente repositorio de GitHub: \href{https://github.com/t4dokiary/knn}{https://github.com/t4dokiary/knn}

\subsection{Distancia Euclidiana} \label{subsec:Distancia Euclidiana}

La distancia euclidiana es un número positivo que indica la separación que tienen dos puntos en un
espacio donde se cumplen los axiomas y teoremas de la geometría de Euclides.
La distancia entre dos puntos A y B de un espacio euclidiano es la longitud del vector AB
perteneciente a la única recta que pasa por dichos puntos.
Se define la distancia euclidiana d(A,B) entre los puntos A y B, ubicados sobre una recta, como la
raíz cuadrada del cuadrado de las diferencias de sus coordenadas X, esta definición garantiza que:
la distancia entre dos puntos sea siempre una cantidad positiva. Y que la distancia entre A y B sea
igual a la distancia entre B y A.

