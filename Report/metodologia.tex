% metodologia.tex

\section{Metodologia} \label{sec:Metodologia}

\begin{enumerate}
    \item \textbf{Conversión de Datos Nominales a Numéricos}:
    \begin{itemize}
        \item Utilizaremos la base de datos GOLF y realizaremos la conversión de los datos nominales a datos numéricos.
    \end{itemize}

    \item \textbf{División de la Base de Datos}:
    \begin{itemize}
        \item Dividiremos la base de datos en dos subconjuntos: conjunto de entrenamiento y conjunto de prueba.
        \item El conjunto de entrenamiento tendrá un tamaño de 10 elementos, mientras que el conjunto de prueba tendrá un tamaño de 4 elementos.
        \item La selección de elementos se realizará de forma aleatoria.
    \end{itemize}

    \item \textbf{Clasificación para $K=3$, $K=5$ y $K=7$}:
    \begin{itemize}
        \item Utilizaremos el algoritmo KNN con diferentes valores de $K$ (3, 5 y 7) para clasificar uno de los 4 elementos del conjunto de prueba.
        \item Realizaremos el mismo procedimiento para cada valor de $K$.
    \end{itemize}

    \item \textbf{Cálculo del Porcentaje de Eficiencia}:
    \begin{itemize}
        \item Después de clasificar los elementos del conjunto de prueba para cada valor de $K$, calcularemos el porcentaje de eficiencia de cada clasificación.
    \end{itemize}

    \item \textbf{Herramientas Utilizadas}:
    \begin{itemize}
        \item Para llevar a cabo este estudio, utilizaremos los siguientes módulos:
        \begin{itemize}
            \item \textit{NumPy}: para operaciones numéricas.
            \item \textit{Pandas}: para manipulación y análisis de datos.
            \item \textit{heapq}: para realizar operaciones de clasificación.
        \end{itemize}
        \item Además, convertiremos la base de datos a un archivo CSV para facilitar su manipulación y análisis.
    \end{itemize}


    \item \textbf{Etapas}:
    \begin{itemize}

        \item \textbf{Carga y Preprocesamiento de Datos}:
        \begin{itemize}
            \item Se utiliza la biblioteca Pandas para cargar los datos desde un archivo CSV denominado \texttt{golf.csv}.
            \item La columna \texttt{"Unnamed: 7"}, si está presente, se elimina para evitar posibles conflictos en el análisis.
        \end{itemize}

        \item \textbf{Manipulación de Datos con NumPy}:
        \begin{itemize}
            \item Los datos cargados se convierten a un formato NumPy array para facilitar su manipulación y cálculos numéricos posteriores.
            \item Se mezclan aleatoriamente los datos para evitar sesgos en el análisis y garantizar la aleatoriedad en la selección de datos de entrenamiento y prueba.
        \end{itemize}

        \item \textbf{División de Datos}:
        \begin{itemize}
            \item Se dividen los datos en dos conjuntos: un conjunto de entrenamiento y un conjunto de prueba.
            \item El conjunto de entrenamiento contiene 10 instancias, mientras que el conjunto de prueba contiene 4 instancias.
            \item Se elimina la columna de etiquetas de clase del conjunto de entrenamiento y prueba, ya que solo se necesita para la clasificación.
        \end{itemize}

        \item \textbf{Aplicación del Algoritmo KNN}:
        \begin{itemize}
            \item Se solicita al usuario el valor de $k$ que determinará la cantidad de vecinos más cercanos a considerar en el proceso de clasificación.
            \item Para cada instancia en el conjunto de prueba, se calculan las distancias euclidianas respecto a todas las instancias del conjunto de entrenamiento.
            \item Se identifican los $k$ vecinos más cercanos a partir de las distancias calculadas utilizando la estructura de datos \texttt{heapq}.
            \item Se determina la clase de la instancia de prueba mediante un voto mayoritario entre las clases de los vecinos más cercanos.
        \end{itemize}

        \item \textbf{Resultados y Conclusiones}:
        \begin{itemize}
            \item Se muestra el resultado final de la clasificación para cada instancia del conjunto de prueba, indicando si se clasifica como \texttt{"Yes"} o \texttt{"No"} según el voto mayoritario.
            \item El proceso se repite para diferentes valores de $k$ para evaluar la sensibilidad del modelo a este hiperparámetro.
        \end{itemize}

    \end{itemize}
\end{enumerate}
\textbf{BASE DE DATOS GOLF}:
\begin{itemize}
    \item La base de datos GOLF es una base de datos de ejemplo.
\end{itemize}
